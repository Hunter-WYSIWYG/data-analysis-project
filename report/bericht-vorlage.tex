\documentclass[usegeometry=true]{scrartcl}
\usepackage[ngerman]{babel}
\usepackage[T1]{fontenc}
\usepackage{lmodern}
\usepackage[utf8]{inputenc}
\usepackage{hyperref}
\usepackage{amssymb}
% Dimensionen bitte nicht ändern. 
\usepackage[left=2cm, right=2cm, top=2cm, bottom=2cm, bindingoffset=1cm, includeheadfoot]{geometry}
%Zeilenabstand bitte nicht ändern
\usepackage[onehalfspacing]{setspace}

\usepackage[backend=biber,style=numeric,]{biblatex}\addbibresource{literatur.bib}

\begin{document}
% ----------------------------------------------------------------------------
\subject{Projektbericht zum Modul Information Retrieval und Visualisierung Sommersemester 2021}
\title{Titel des Dokuments}
%\subtitle{Untertitel}% optional
\author{Marcus Gagelmann}% obligatorisch
%\date{10.9.2021}
\maketitle% verwendet die zuvor gemachte Angaben zur Gestaltung eines Titels
% ----------------------------------------------------------------------------
% Inhaltsverzeichnis:
%\tableofcontents
% ----------------------------------------------------------------------------
% Gliederung und Text:

\section{Einleitung}
Afrika bildet mit einer Größe von mehr als 30 Millionen Quadratkilometern den zweitgrößten Kontinent der Erde. Insgesamt leben dort über 1,3 Milliarden Menschen, was etwa 17,2\% der Weltbevölkerung ausmacht. Wo jedoch Land und Menschen aufeinander treffen, dort sind auch Konflikte nicht weit entfernt. Seid vielen Jahren fällt Afrika solchen sowohl politischen als auch kriegerischen Auseinandersetzungen zum Opfer. Ursachen hierfür könnten die verbreitete Armut, misswirtschaftende Regierungen sowie die wertvollen Ressourcen des Kontinents sein, um nur einige mögliche Gründe zu nennen. Seit 1997 werden diese Konflikte durch das ACLED-Projekt dokumentiert. ACLED steht hierbei für \glqq Armed Conflict Location and Event Data\grqq. Die gesammelten Daten werden von ACLED auf ihrer Website (\url{https://acleddata.com/#/dashboard}) frei zugänglich zur Verfügung gestellt.\\

Nach nun beinahe 25 Jahren an Konfliktaufzeichnungen sind mittlerweile sehr große Datenmengen entstanden, welche einem außenstehenden Tabellenbetrachter kaum einen Überblick über das gesamtgeschehen seit dem Aufzeichnungsbeginn geben können. Noch unwahrscheinlicher ist es, dass ein solcher Betrachter allein mithilfe der unaufbereiteten Daten Rückschlüsse auf mögliche Zusammenhänge zwischen den vielen Konflikten ziehen kann. Mit dieser Ausgangslage ist ein Verstehen der tatsächlichen Lage in Afrika anhand der Daten unmöglich. Hierdurch könnten mögliche Maßnahmen zur Verbesserung der Situation des Kontinents weniger effektiv ausfallen, als wenn das volle Potenzial der Daten ausgeschöpft werden würde.\\

Aus diesem Problem ergibt sich die Fragestellung, ob der gewählte Datensatz hilfreiche Aussagen zu Zusammenhängen ermöglicht, welche die Opfer der Konflikte der letzten 25 Jahre betreffen. Das Ziel dieser Arbeit ist deshalb die Bereitstellung einer Anwendung, welche eine Analyse der Todesopfer der Afrikakonflikte ermöglicht. Mithilfe dieser Anwendung werden einzelne Konflikte für einen Gesamtüberblick über eine oder mehrere Regionen Afrikas zusammengefasst, sowie nach einzelnen Jahren sortiert und hierbei auch mehrdimensional visualisiert.

\subsection{Anwendungshintergrund}
Sie müssen genug Hintergrund bereitstellen, so dass die Lesenden sich ein Urteil bilden können, ob ihre Lösung funktioniert. Sie sollen die Lesenden jedoch nicht mit Anwendungsdetails so überschütten, dass der Fokus auf die Fragen zur Informationsvisualisierung untergehen. 
\subsection{Zielgruppen}
Beschreiben sie die Personengruppe oder Personengruppen, die das von ihnen benannte Anwendungsproblem lösen möchte. Auf welches Vorwissen können sie in dieser Gruppen von Anwenderinnen aufbauen? Welche Informations"-bedürf"-nisse werden durch die Visualisierungen adressiert?
\subsection{Überblick und Beiträge}
In diesem Abschnitt geben sie einen kurzen Überblick über die Daten und verwendeten Visualisierungen. Dann benennen sie die Beiträge ihres Projekts. Diese Beiträge müssen sie in den hinteren Teilen des Berichts genauer ausführen und belegen.

\section{Daten}
Beschreiben Sie vorhandenen Daten. Gehen sie kritisch darauf ein, in wie weit sich die Daten für die Bearbeitung der Fragestellungen und dem Erreichen von Lösungen für die oben beschriebene Zielgruppen eignen. Haben sie die Daten sinnvoll mit weiteren Datenquellen ergänzt? Wenn ja, wie?
\subsection{Technische Bereitstellung der Daten}
Wie sind die Daten zugänglich? Welche Formate werden genutzt. Gibt es Besonderheiten beim Lesen der Formate?
\subsection{Datenvorverarbeitung}
Welche Datenvorverarbeitungsschritte sind notwendig? Beschreiben Sie die einzelnen Schritte und begründen sie sie, z.B. warum werden manche Daten weggelassen, über welche Mengen werden Durchschnitte berechnet, warum sind die so berechneten Werte aussagekräftiger als andere Werte. 

\section{Visualisierungen}
\subsection{Analyse der Anwendungsaufgaben}
Analysieren sie die konkreten Anwendungsaufgaben. Welche Visualisierungen helfen den Personen, die die Software verwenden, sinnvolle mentale Modelle aufzubauen. Sind diese mentalen Modelle für sie notwendig, um die Aufgaben lösen zu können?
\subsection{Anforderungen an die Visualisierungen}
Leiten sie Anforderungen an das Design der Visualisierungen ab, die sich durch ihre Analyse des Zielproblems ergeben.
\subsection{Präsentation der Visualisierungen}
Präsentieren sie die visuelle Abbildungen und Kodierungen der Daten und Interaktionsmöglichkeiten. 
Sie müssen  begründen, warum und wiegut ihre Designentscheidungen die erstellten Anforderungen erfüllen. 
Weiterhin müssen sie begründen, warum die gewählte visuelle Kodierung der Daten für das zulösenden Problem passend ist. 
Typische Argumente würden hier auf Wahrnehmungsprinzipien und Theorie über Informationsvisualisierung verweisen. 
Die besten Begründungen diskutieren explizit die konkrete Auswahl der Visualisierungen im Kontext von mehreren verschiedenen Alternativen. Diskutieren sie die Expressivität und die Effektivität der einzelnen Visualisierungen. 

Die eben beschriebenen Präsentationen und Begründungen sollen für jede der drei folgenden Visualisierungen durchgeführt werden. 
\subsubsection{Visualisierung Eins}
\subsubsection{Visualisierung Zwei}
\subsubsection{Visualisierung Drei}

\subsection{Interaktion}
Erklären sie die möglichen Interaktionen mit den einzelnen Visualisierungen und die möglichen Verknüpfungen zwischen ihnen. Begründen Sie warum die konkreten Interaktionen umgesetzt wurden und welche Zwecke für die Anwenderinnen mit ihnen unterstützt werden. Begründen sie ebenfalls warum sie andere Interaktionsmöglichkeiten nicht umgesetzt haben. 

\section{Implementierung}
Beschreiben Sie die Implementierung ihrer Visualisierungsanwendung in Elm. Stellen die Gliederung ihres Quellcodes vor. Haben Sie verschiedene Elm-Module erstellt. Was war aufwändig umzusetzen, was ließ sich mit dem vorhanden Code aus den Übungen relativ einfach umsetzen? 

Wie sieht die Elm-Datenstruktur für das Model aus, in dem die verschiedenen Zustände der Interaktion gespeichert werden können.

\section{Anwendungsfälle}
Präsentieren sie für jede der drei Visualisierungen einen sinnvollen Anwendungsfall in dem ein bestimmter Fakt, ein Muster oder die Abwesenheit eines Musters visuell festgestellt wird. Begründen sie warum dieser Anwendungsfall wichtig für die Zielgruppe der Anwenderinnen ist. Diskutieren sie weiterhin, ob die oben beschriebene Information auch mit anderen Visualisierungstechniken hätte gefunden werden können. Falls dies möglich wäre, vergleichen sie die den Aufwand und die Schwierigkeiten ihres Ansatzes und der Alternativen. 
\subsection{Anwendung Visualisierung Eins}
\subsection{Anwendung Visualisierung Zwei}
\subsection{Anwendung Visualisierung Drei}

\section{Verwandte Arbeiten}
Führen sie eine kurze Literatursuche in der wissenschaftlichen Literatur zu Informationsvisualisierung und Visual Analytics nach ähnlichen Anwendungen durch. Diskutieren sie mindestens zwei Artikel. Stellen sie Gemeinsamkeiten und Unterschiede dar.

\section{Zusammenfassung und Ausblick}
Fassen sie die Beiträge ihre Visualisierungsanwendung zusammen. Wo bietet sie für die Personen der Zielgruppe einen echten Mehrwert.

Was wären mögliche sinnvolle Erweiterungen, entweder auf der Ebene der Visualisierungen und/oder auf der Datenebene?

\section*{Anhang: Git-Historie}

\printbibliography

\end{document}

